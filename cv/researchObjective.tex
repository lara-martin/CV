{\color{black}\fontsize{12pt}{1em}\faIcon{search}} \cvsection{Research Interests}

%\entrytitlestyle{Human-Centered AI}

%generally through neurosymbolic methods

\headerquotestyle{Human-Centered Artificial Intelligence \& Natural Language Processing, Computational Creativity, Automated Story Generation \& Understanding, Tabletop Roleplaying Agents, Neurosymbolic Methods, Augmentative \& Alternative Communication (AAC), Speech Processing, Affective Computing, Conversational Agents, Cognitive Systems}


%Data scientist, educator, sysadmin, and data security specialist with over nine years designing and delivering technical solutions for academic and student research projects at the Macquarie University Faculty of Arts and UNSW Australia. Chief Investigator in category one and two grants, contracts, and prizes across the humanities, social sciences, and security studies totalling over AUD\$3,248,624. Lead Investigator on a big data investigation of violent extremism using tens of millions of posts on social media deploying computational data collection and analysis techniques using Machine Learning and Natural Language Processing. Presented findings for this project at the highest levels of state and federal governments, leading to international collaborations with NATO and the United States Institute for Peace. Technical Director, Product Owner, and DevOps/SysAdmin for a field-data collection project, delivering 64+ field data collection modules since 2013. Educator, trainer, and mentor staff and postgraduate students across the Faculty of Arts conducting digitally-enabled research. Trainer and Instructor for The Carpentries, leading over eleven workshops in data management, reproducible research techniques, and pedagogical techniques. Convener of undergraduate and postgraduate units in Information Technology and Digital Humanities topics. Supervised two MRes students applying and assessing Machine Learning techniques in sociology and ancient history. Strong industry ties with Google, Ubisoft, and Twitter -- bringing Macquarie University on as the first university on Google Arts and Culture, invited to be an academic advisor for the Fabricius Workbench, and facilitating big data social media collection on Twitter. Provision of technical advice, ethics and security advice, data management plans, database designs, and solution-focused support for academic and student research projects across Australian and International academic institutions. An internationally respected researcher in the Digital Humanities with sixteen peer-reviewed publications, commissioned reports, newspaper articles, book chapters, and conference presentations.



