%!TEX TS-program = xelatex
%!TEX encoding = UTF-8 Unicode
% Awesome CV LaTeX Template for CV/Resume
% 
% Template license:
% CC BY-SA 4.0 (https://creativecommons.org/licenses/by-sa/4.0/)
%
% Adaptation by
% Lara J. Martin (laramar@seas.upenn.edu)
%
% Template from 
% https://github.com/Denubis/Academic-LaTeX-CV-BibTeX-CSV
% Brian Ballsun-Stanton (brian.ballsun-stanton@mq.edu.au)
%
% Originally taken from
% https://github.com/posquit0/Awesome-CV
% Claud D. Park <posquit0.bj@gmail.com>
% http://www.posquit0.com
%-------------------------------------------------------------------------------
% CONFIGURATIONS
%-------------------------------------------------------------------------------
% A4 paper size by default, use 'letterpaper' for US letter
\documentclass[11pt, a4paper]{awesome-cv}


\usepackage{datatool}
\usepackage{xparse}
\usepackage{eso-pic,graphicx}
\usepackage{pstricks}
\usepackage{hyphenat}

    
% https://tex.stackexchange.com/a/376174

% Configure page margins with geometry
\geometry{left=1.4cm, top=.8cm, right=1.4cm, bottom=1.8cm, footskip=.5cm}

% Specify the location of the included fonts
\fontdir[fonts/]

% Color for highlights
% Awesome Colors: awesome-emerald, awesome-skyblue, awesome-red, awesome-pink, awesome-orange
%                 awesome-nephritis, awesome-concrete, awesome-darknight
%\colorlet{awesome}{awesome-red}
% Uncomment if you would like to specify your own color
% \definecolor{awesome}{HTML}{CA63A8}

% Colors for text
% Uncomment if you would like to specify your own color
% \definecolor{darktext}{HTML}{414141}
% \definecolor{text}{HTML}{333333}
% \definecolor{graytext}{HTML}{5D5D5D}
% \definecolor{lighttext}{HTML}{999999}

% Set false if you don't want to highlight section with awesome color
\setbool{acvSectionColorHighlight}{false}

% If you would like to change the social information separator from a pipe (|) to something else
\renewcommand{\acvHeaderSocialSep}{\quad\textbar\quad}

\newcommand{\myname}{\textbf{L. J. Martin}}
\newcommand{\mynamedagger}{\textbf{L. J. Martin$^\dagger$}}

% Following entry pattern: https://zety.com/blog/academic-cv-example


%-------------------------------------------------------------------------------
%	PERSONAL INFORMATION
%	Comment any of the lines below if they are not required
%-------------------------------------------------------------------------------
% Available options: circle|rectangle,edge/noedge,left/right
% \photo{./examples/profile.png}

\addbibresource{bibliography.bib}

%%%%%%%%%%%%%%%%%%%%%%
% Sort Bibliography
% https://tex.stackexchange.com/questions/484108/biblatex-sort-by-year-and-name-both-descending
\DeclareSortingTemplate{ydndt}{
  \sort{
    \field{presort}
  }
  \sort[final]{
    \field{sortkey}
  }
  \sort[direction=descending]{
    \field{sortyear}
    \field{year}
    \literal{9999}
  }
  \sort[direction=descending]{
    \field{sortmonth}
    \field{month}
    \literal{12}
  }
  \sort{
    \field{sorttitle}
    \field{title}
  }
}
%%%%%%%%%%%%%%%%%%%%%%
% Sort citation numbers
% https://tex.stackexchange.com/questions/273313/biblatex-sort-cites-via-number-split-bibliography
% https://tex.stackexchange.com/questions/310684/how-to-get-sorted-and-compressed-citations-when-using-split-bibliography-lists-a
\DeclareSourcemap{
  \maps[datatype=bibtex]{
    \map{
     \step[fieldsource=keywords, match=thesis, final]
     \step[fieldset=presort, fieldvalue = {A}]
    }
    \map{
      \step[fieldsource=keywords, match=conference, final]
      \step[fieldset=presort, fieldvalue = {B}]
    }
    \map{
      \step[fieldsource=keywords, match=workshop, final]
      \step[fieldset=presort, fieldvalue = {C}]
    }
    \map{
      \step[fieldsource=keywords, match=other, final]
      \step[fieldset=presort, fieldvalue = {D}]
    }
    \map{
      \step[fieldsource=keywords, match=preprint, final]
      \step[fieldset=presort, fieldvalue = {E}]
    }
    \map{
      \step[fieldsource=keywords, match=submitted, final]
      \step[fieldset=presort, fieldvalue = {F}]
    }
  }
}
%%%%%%%%%%%%%%%%%%%%%%




% Available options: circle|rectangle,edge/noedge,left/right
% \photo{./examples/profile.png}
\name{Dr.}{Lara J.}{Martin}
\position{CIFellow Postdoctoral Researcher \bullet~University of Pennsylvania}
\address{200 South 33rd Street, Philadelphia PA, 19104}

%\mobile{(+1) 609-975-9799}
\homepage{laramartin.net}
\email{laramar@seas.upenn.edu}
%\github{lara-martin}
\linkedin{lara-j-martin}
% \gitlab{gitlab-id}
% \stackoverflow{SO-id}{SO-name}
\twitter{@LangTechLara}
% \skype{skype-id}
% \reddit{reddit-id}
% \medium{madium-id}
\googlescholar{YjiWURYAAAAJ}{Google Scholar Page}
%% \firstname and \lastname will be used
% \googlescholar{googlescholar-id}{}
% \extrainfo{extra informations}

%\quote{``The purpose of a system is what it does." - Stafford Beer} But quotes at the start of CVs seems... tacky. 


%-------------------------------------------------------------------------------
\begin{document}

\AddToShipoutPictureBG*{%
  \AtPageUpperLeft{%
    \raisebox{-\height}{%
      \includegraphics[width=\paperwidth,height=0.175\paperwidth]{sidebar.png}%
    }%
  }
}

% Print the header with above personal information
% Give optional argument to change alignment(C: center, L: left, R: right)
\makecvheader[C]



% Print the footer with 3 arguments(<left>, <center>, <right>)
% Leave any of these blank if they are not needed
\makecvfooter
  {Last updated: \today}
  {Dr. Lara J. Martin~~~·~~~Curriculum Vitae}
  {\thepage}


%-------------------------------------------------------------------------------
%	CV/RESUME CONTENT
%	Each section is imported separately, open each file in turn to modify content
%-------------------------------------------------------------------------------
\vspace{1cm}
\input{cv/researchObjective.tex}
\vspace{0.5cm}
%-------------------------------------------------------------------------------
%	SECTION TITLE
%-------------------------------------------------------------------------------
{\color{black}\fontsize{12pt}{1em}\faIcon{graduation-cap}} \cvsection{ Education}


%-------------------------------------------------------------------------------
%	CONTENT
%-------------------------------------------------------------------------------
\begin{cventries}

%---------------------------------------------------------
  \cventry
    {Ph.D. in Human-Centered Computing} % Degree
    {Georgia Institute of Technology} % Institution
    {Aug. 2015 - May 2021} %– Dec. 2020} % Date(s) -- May 7
    {Atlanta, GA} % Location
    {
      \begin{cvitems} % Description(s) bullet points
       % \item{School of Interactive Computing, College of Computing}
        \item {Advisor: Dr. Mark O. Riedl}
        \item {Thesis: \href{https://smartech.gatech.edu/handle/1853/64643}{Neurosymbolic Automated Story Generation}}
        \item{Teaching Certification: \href{https://ctl.gatech.edu/tech-teaching}{Tech to Teaching} (Summer 2018)}
      \end{cvitems}
    }



 \cventry
    {M.S. in Language Technologies} % Degree
    {Carnegie Mellon University} % Institution
    {Aug. 2013 – May 2015} % Date(s) 5/17 was commencement
    {Pittsburgh, PA} % Location
    {      
      \begin{cvitems} % Description(s) bullet points
        \item {Advisor: Dr. Alan W Black}
      \end{cvitems}
    }


  \cventry
    {B.S. in Computer Science \& Linguistics (double major)} % Degree
    {Rutgers University --- New Brunswick} % Institution
    {Sep. 2009 – May 2013} % Date(s) 5/19/2013
    {Piscataway, NJ} % Location
    {      
      \begin{cvitems} % Description(s) bullet points
      \item {Advisor: Dr. Matthew Stone}
      \end{cvitems}
    }%
%---------------------------------------------------------
\end{cventries}

\vspace{0.5cm}
%-------------------------------------------------------------------------------
%	SECTION TITLE
%-------------------------------------------------------------------------------
{\color{black}\fontsize{12pt}{1em}\faIcon{building}} \cvsection{Research Experience}


%-------------------------------------------------------------------------------
%	CONTENT
%-------------------------------------------------------------------------------
\begin{cventries}


%---------------------------------------------------------
  \cventry
    {University of Maryland, Baltimore County -- Computer Science and Electrical Engineering (CSEE) Department} % Organization
    {Assistant Professor} % Job title
    {Aug 2023 – Present} % Date(s)
    {Baltimore, MD} % Location
    {
      % \begin{cvitems} % Description(s) of tasks/responsibilities
      % \end{cvitems}
    }

%---------------------------------------------------------
  \cventry
    {University of Pennsylvania -- Computer and Information Science Department} % Organization
    {\href{https://cifellows2020.org/}{Computing Innovation Fellow (CIFellow)} Postdoctoral Researcher} % Job title
    {Jan 2021 – Aug 2023} % Date(s)
    {Philadelphia, PA} % Location
    {
      % \begin{cvitems} % Description(s) of tasks/responsibilities
      % \item {Identifying the story understanding capabilities of large language models (LLMs).}
      % \item {Developing a working AAC prototype given feedback from users.}
      % \item {Conducted semi-structured interviews with autistic adult users of augmentative and alternative communication (AAC).}
      % \end{cvitems}
    }
%---------------------------------------------------------
  \cventry
    {Georgia Institute of Technology -- School of Interactive Computing} % Organization
    {Graduate Research Assistant} % Job title
    {Aug 2015 – Dec 2020} % Date(s)
    {Atlanta, GA} % Location
    {
      % \begin{cvitems} % Description(s) of tasks/responsibilities
      %   \item {Created a complex end-to-end automated story generation pipeline.}
      % \end{cvitems}
    }

%---------------------------------------------------------
  \cventry
    {Amazon.com Inc. -- Alexa Smart Home Machine Learning} % Organization
    {Applied Scientist Intern} % Job title
    {May 2017 – Aug 2017} % Date(s)
    {Seattle, WA} % Location
    {
      % \begin{cvitems} % Description(s) of tasks/responsibilities
      %   \item {Identified potential research questions within Alexa Smart Home.}
      %   \item {Developed a system for identifying commands with an assumed context.}
      % \end{cvitems}
    }

%---------------------------------------------------------
  \cventry
   {Carnegie Mellon University -- Language Technologies Institute} % Organization
    {Graduate Research Assistant} % Job title
    {Sept 2013 – Aug 2015} % Date(s)
    {Pittsburgh, PA} % Location
    {
      % \begin{cvitems} % Description(s) of tasks/responsibilities
      %   \item {Created a zero-resource speech-to-speech translation system for the University of Pittsburgh Medical Center.}
      %   \item {Performed emotion recognition in noisy speech for event detection.}
      % \end{cvitems}
    }
%---------------------------------------------------------
  \cventry
   {University of Southern California -- Institute for Creative Technologies} % Organization
    {Intern} % Job title
    {May 2011 – Aug 2011} % Date(s)
    {Playa Vista, CA} % Location
    {
      % \begin{cvitems} % Description(s) of tasks/responsibilities
      %   \item {Wrote a chatbot for the Virtual Patient Project using Bruce Wilcox’s language Chatscript.}
      %   \item {Developed an authoring tool for the Chatscript language using Java.}
      %   \item {Designed and ran experiments comparing my Chatscript system to the project’s current chat system.}
      % \end{cvitems}
    }

%---------------------------------------------------------
\end{cventries}

\vspace{.5cm}




%\vspace{-10pt}

%{\color{black}\fontsize{12pt}{1em}} 
\cvsection{\faIcon{chalkboard-teacher}}{Teaching}

%\vspace{4mm}

% \begin{cventries}
% \DTLforeach{teaching}{%
% %ProjectName    Date    Role    Location    Description1 Link  
% \projectname=ProjectName,%
% \date=Date,%
% \role=Role,%
% \location=Location,%
% \descone=Description1,%
% \link=Link%
% }%
% {%
%     \DTLifnullorempty{\link}
%     {
%     \cventryNoNote
%     {\role} % subhead
%     {\projectname} % head
%     {\date} % green
%     {\location} % flushright grey
%     }%
%     {%
%     \cventryteach
%     {\role} % subhead
%     {\href{\link}{\projectname}} % head
%     {\date} % green
%     {\location} % flushright grey
%     {%
%     \DTLifnullorempty{\descone}{}
%     {
%     \begin{cvitems}
%     \item \descone%
%     \end{cvitems}}
%     }%
%     }%
% }
% \end{cventries}

\DTLloaddb[]{teaching}{cv/teaching.csv}

\newcommand*{\uniqueclasses}{}
\newcommand*{\uniquelocation}{}

\DTLforeach{teaching}{%
\classtitle=ProjectName,%
\date=Date,%
\role=Role,%
\location=Location,%
\descone=Description1,%
\link=Link%
}%
{%
  \expandafter\DTLifinlist\expandafter{\location}{\uniquelocation}%
  {}% do nothing, already in list
  {% add to list  
    \ifdefempty{\uniquelocation}%
    {\let\uniquelocation\location}% first element of list
    {% append to list
      \eappto\uniquelocation{,\location}%
    }%
  }%
}

\psforeach{\classlocation}{\uniquelocation}{% for each unique location


\cvsubsection{\ifthenelse{\equal{\classlocation}{University of Maryland Baltimore County}}{University of Maryland, Baltimore County}{\classlocation}}%

    \DTLforeach[\equal{\location}{\classlocation}]{teaching}{% for each class, if it matches the location 
        \projectname=ProjectName,%
        \d=Date,%
        \r=Role,%
        \location=Location,%
        \d=Description1,%
        \l=Link%
        }%
        {%
            \expandafter\DTLifinlist\expandafter{\projectname}{\uniqueclasses}%
            {}%
            {%                  
                \begin{tabular*}{\textwidth}{@{\extracolsep{\fill}} L{\textwidth - 5cm}}%
                \nohyphens{\entrytitlestyle{\projectname}}%
                \end{tabular*}%
                \newline%            
                \ifdefempty{\uniqueclasses}%
                {\let\uniqueclasses\projectname}% first element of list
                {\eappto\uniqueclasses{,\projectname}}% append to list
                \DTLforeach[\equal{\projectname}{\name}]{teaching}%
                    {% for each class, if it matches the title 
                    \name=ProjectName,%
                    \date=Date,%
                    \role=Role,%
                    \loc=Location,%
                    \descone=Description1,%
                    \link=Link%
                    }%
                    {% 
                        \hspace{3mm}\entrylocationstyle{\href{\link}{\date}} -- \entrypositionstyle{\role}%
                        \DTLifnullorempty{\descone}{}{\subdescriptionstyle{, \descone}}%
                        \newline%
                    }%
            }%
        }% 
        \vspace{-10pt}
}%




\vspace{.5cm}
{\color{black}\fontsize{12pt}{1em}\faIcon{box}} \cvsection{Projects}

% \DTLloaddb[]{projects}{cv/projects.csv}

% \cvsubsection{Summary}
% \def\projectaffiliated{0}
% \def\projectexternal{0}
% \def\projecttotal{0}

% \DTLforeach{projects}{%
% %GrantAffiliated	External
% \affiliated=GrantAffiliated,%
% \external=External}{%
% \DTLgadd{\projectaffiliated}{\affiliated}{\projectaffiliated}%
% \DTLgadd{\projectexternal}{\external}{\projectexternal}%
% \DTLgadd{\projecttotal}{1}{\projecttotal}%
% }
% \begin{cvhonors}
% \cvhonor{projects with external funding}{excluding FAIMS sub-projects}{2012 -- Present}{\projectaffiliated{}}
% \cvhonor{projects with external impact}{excluding FAIMS sub-projects}{2012 -- Present}{\projectexternal{}}
% \cvhonor{total projects}{excluding FAIMS sub-projects}{2012 -- Present}{\projecttotal{}}
% \cvhonor{FAIMS sub-projects}{modules, extensions, and custom exporters}{2014 -- Present}{64}
% \end{cvhonors}
% \cvsubsection{Project Details}
% \begin{cventries}
% \DTLforeach{projects}{%
% %ProjectName	StartYear	StopYear	Collaborator	Tools	Description1	Description2
% \projectname=ProjectName,%
% \startyear=StartYear,%
% \stopyear=StopYear,%
% \collaborator=Collaborator,%
% \tools=Tools,%
% \descone=Description1,%
% \desctwo=Description2}%
% {%
%  \cventry
%     {\tools} % subhead
%     {\projectname} % head
%     {\startyear{} -- \DTLifnullorempty{\stopyear}{Present}{\stopyear{}}} % green
%     {\collaborator} % flushright grey
%     {%
%     %\ifthenelse{\DTLcurrentindex=16}{\dtlbreak}{}%
%     \DTLifnullorempty
%         {\descone}
%         {}
%         {\begin{cvitems}
%         \item \descone%
%         \DTLifnullorempty
%             {\desctwo}
%             {}
%             {\item \desctwo}
%         \end{cvitems}
%         }
%     }
% }



% \end{cventries}

{\color{teal}\faIcon{icons} {\bf Human-AI Communication and Computational Creativity.}} {\bodyfontlight Collaboration between people and AI, primarily through storytelling.}

\begin{tabular*}{\textwidth}{@{\extracolsep{\fill}}p{.2cm} L{6cm} L{\textwidth - 10cm} @{\extracolsep{\fill}}R{3cm}}
    \setlength{\extrarowheight}{5pt}
    \faIcon{book-open} & Story understanding &  \cite{Li2022,Giorgi2023,Dong2023} & \entrylocationstyle{2021 -- Present}\\
    \faIcon{d-and-d} & D\&D as an AI challenge & \cite{Zhu2023AIIDE,martin2018dungeons,CallisonBurch2022,Papazov2022, CallisonBurchEMNLP,Zhu2023} & \entrylocationstyle{2018 -- Present}\\
    \faIcon{edit} & Story generation &  \cite{martin2016improvisational,martin2017improvisational,martin2017event,martin2018event,ammanabrolu2019guided,tambwekar2019controllable,ammanabrolu2020story,martin2021thesis,Alabdulkarim2021} & \entrylocationstyle{2015 -- Present}\\
    \faIcon{bezier-curve} & Schema co-creation & \cite{Zhang2023} & \entrylocationstyle{2023}\\
    \faIcon{comments} & Conversational agents & \cite{panagopoulouquakerbot}, See also: Amazon \& USC ICT internships & \entrylocationstyle{2011, 2017, 2022}\\
    %mobile 
    \faIcon{headphones} & Language learning & \cite{wolfeapplause} & \entrylocationstyle{2014}\\
    
\end{tabular*}

\vspace{.1cm}

{\color{teal}\faIcon{people-arrows} {\bf Human-Human Communication.}} {\bodyfontlight Analysis of human-human communication or computer-mediated communication.}



\begin{tabular*}{\textwidth}{@{\extracolsep{\fill}} p{.2cm} L{9cm} L{5.6cm} R{2.2cm}}
    \setlength{\extrarowheight}{5pt}
    \faIcon{tty} & Augmentative and Alternative Communication (AAC) & In progress. & \entrylocationstyle{2022 -- Present}\\
    \faIcon{reddit-alien} & Online communities  &\cite{moon2014identifying,Giorgi2023} & \entrylocationstyle{2014, 2023}\\
    \faIcon{language} & Translation & \cite{martin2015utterance}& \entrylocationstyle{2015}\\
    \faIcon{frown-open} & Emotion recognition \& affective computing &  \cite{martin2014methodology,cosentino2014,Yu2014,Yu2015}& \entrylocationstyle{2013 -- 2015}\\
    
    
\end{tabular*}


\vspace{.5cm}
%-------------------------------------------------------------------------------
%	SECTION TITLE
%-------------------------------------------------------------------------------
%{\color{black}\fontsize{12pt}{1em}} 
\cvsection{\faIcon{paper-plane}}{Publications}
\vspace{0.5em}
\defbibnote{bothnote}{ {\footnotesize $^*$equal contribution, $^\dagger$presented}}
\defbibnote{presented}{ {\footnotesize $^\dagger$presented}}

\nocite{*}
\printbibliography[keyword={thesis}, title=PhD Dissertation]
\vspace{10pt}
\printbibliography[prenote=bothnote, keyword={conference}, title=Conference Proceedings]
\vspace{10pt}
\printbibliography[prenote=bothnote, keyword={workshop}, title=Refereed Workshop Papers]
\vspace{10pt}
\printbibliography[keyword={other}, title=Other Publications]
\vspace{10pt}
\printbibliography[keyword={preprint}, title=Preprints]
\vspace{10pt}
\printbibliography[keyword={submission}, title=In Submission]




\vspace{.5cm}
%grants
{\color{black}\fontsize{12pt}{1em}\faIcon{users}} \cvsection{Institutional Service}

\begin{cventries}
\cventry
    {Reviewer}
    {President's Undergraduate Research Awards (PURA)}
    {Summer 2019}
    {Georgia Institute of Technology}
    {}    
\cventry
    {Volunteer}
    {School of Interactive Computing’s Prospective Student Visit Week}
    {Spring '16, '17, '18}
    {Georgia Institute of Technology}
    {}
\cventry
    {Member}
    {School of Interactive Computing Faculty Hiring Committee}
    {Fall 2018}
    {Georgia Institute of Technology}
    {}
\cventry
    {Member}
    {Graduate Student Council}
    {Fall 2018 -- Spring 2019}
    {Georgia Institute of Technology}
    {}    
\cventry
    {Website Manager}
    {Human-Centered Computing Website}
    {Fall 2017 -- Spring 2019}
    {Georgia Institute of Technology}
    {}    
\cventry
    {Coordinator}
    {School of Interactive Computing's Prospective Student Visit Week}
    {Spring 2016}
    {Georgia Institute of Technology}
    {}
\end{cventries}



\vspace{.5cm}
%-------------------------------------------------------------------------------
%	SECTION TITLE
%-------------------------------------------------------------------------------

{\color{black}\fontsize{12pt}{1em}\faIcon{chart-bar}} \cvsection{Professional Activities - Presentations}

\DTLloaddb[]{presentations}{cv/presentations.csv}
\cvsubsection{Summary}
%\def\invitedpres{0}
\def\campuspres{0}
\def\conferencepres{0}
\def\lecture{0}
\def\totalpres{0}


\DTLforeach{presentations}{%
\status=Status}{%
\DTLifeq{\status}{Campus}{%
\DTLgadd{\campuspres}{1}{\campuspres}%
}{}%
\DTLifeq{\status}{Conference}{%
\DTLgadd{\conferencepres}{1}{\conferencepres}%
}{}%
\DTLifeq{\status}{Guest Lecture}{%
\DTLgadd{\lecture}{1}{\lecture}%
}{}%
}
\DTLsavelastrowcount{\totalpres}


\begin{cvhonors}
\cvhonor{On-Campus Invited Speaker Presentations}{}{2013 -- Present}{\campuspres{}}
\cvhonor{Special Conference Presentations}{}{2019 -- Present}{\conferencepres{}}
\cvhonor{Guest Lectures}{}{2021 -- Present}{\lecture{}}
\cvhonor{Total Presentations}{}{2013 -- Present}{\totalpres{}}
\end{cvhonors}

\newcommand*{\uniquetalks}{}

\DTLforeach[\equal{\acadstatus}{Campus}]{presentations}{%
\prestitle=ProjectName,%
\date=Date,%
\role=Role,%
\location=Location,%
\address=address,%
\acadstatus=Status,%
\link=link,%
\note=note}%
{%
  \expandafter\DTLifinlist\expandafter{\prestitle}{\uniquetalks}%
  {}% do nothing, already in list
  {% add to list
    \ifdefempty{\uniquetalks}%
    {\let\uniquetalks\prestitle}% first element of list
    {% append to list
      \eappto\uniquetalks{,\prestitle}%
    }%
  }%
}

\hspace{2cm}


\cvsubsection{On-Campus Invited Speaker Presentations}


\psforeach{\prestitle}{\uniquetalks}{% for each unique talk title
    \begin{tabular*}{\textwidth}{@{\extracolsep{\fill}} L{\textwidth - 5cm} R{5cm}}
    \nohyphens{\entrytitlestyle{``\prestitle''}}%
    \end{tabular*}%
    \newline
    \DTLforeach[\equal{\projectname}{\prestitle}\AND\equal{\acadstatus}{Campus}]{presentations}{% for each presentation, if it matches the title AND it is a campus talk
    \projectname=ProjectName,%
    \date=Date,%
    \role=Role,%
    \location=Location,%
    \address=address,%
    \acadstatus=Status,%
    \link=link,%
    \note=note}%
    {%
     \noindent\presentry% Usage: \cvsubentry{<position>}{<title>}{<date>}{<description>}
        {\date} % green
        {\href{\link}{\location}} % head
        {\address}% % flushright grey
        {\role}%
    }
}






\cvsubsection{Special Conference Presentations}


\DTLforeach[\equal{\acadstatus}{Conference}]{presentations}{%
\projectname=ProjectName,%
\date=Date,%
\role=Role,%
\location=Location,%
\address=address,%
\acadstatus=Status,%
\link=link,%
\note=note}%
{%
    \begin{tabular*}{\textwidth}{@{\extracolsep{\fill}} L{\textwidth - 5cm} R{5cm}}
     \nohyphens{\entrytitlestyle{``\href{\link}{\projectname}''}}
    \end{tabular*}
    \newline
     \presentry% Usage: \cvsubentry{<position>}{<title>}{<date>}{<description>}
        {\date} % green
        {\location} % head
        {\address}% % flushright grey
        {}%
        \DTLifnullorempty
        {\note}
        {}
        {
        \begin{cvtalknotes}
            {\item {\bf \note}}
        \end{cvtalknotes}
        }
}





\hspace{2cm}

\cvsubsection{Guest Lectures}




\DTLforeach[\equal{\acadstatus}{Guest Lecture}]{presentations}{%
\projectname=ProjectName,%
\date=Date,%
\role=Role,%
\location=Location,%
\address=address,%
\acadstatus=Status,%
\link=link,%
\note=note}%
{%
    \begin{tabular*}{\textwidth}{@{\extracolsep{\fill}} L{\textwidth - 5cm} R{5cm}}
     \nohyphens{\entrytitlestyle{``\href{\link}{\projectname}''}}
    \end{tabular*}
    \newline
    \presentry% Usage: \cvsubentry{<position>}{<title>}{<date>}{<description>}
        {\date} % green
        {\location} % head
        {\address}% % flushright grey
        {}%
}



%-------------------------------------------------------------------------------
%	SECTION TITLE
%-------------------------------------------------------------------------------

{\color{black}\fontsize{12pt}{1em}\faIcon{comments}} \cvsection{Professional Activities -- Conference Organization}

\DTLloaddb[]{organizing}{cv/organizing.csv}
\cvsubsection{Summary}
\def\organizer{0}
\def\chair{0}
\def\reviewer{0}
\def\attendee{0}
\def\totalconf{0}


\DTLforeach{organizing}{%
\status=GeneralRole}{%
\DTLifeq{\status}{Organizer}{%
\DTLgadd{\organizer}{1}{\organizer}%
}{}%
\DTLifeq{\status}{Action Editor}{%
\DTLgadd{\chair}{1}{\chair}%
}{}%
\DTLifeq{\status}{Chair}{%
\DTLgadd{\chair}{1}{\chair}%
}{}%
\DTLifeq{\status}{Reviewer}{%
\DTLgadd{\reviewer}{1}{\reviewer}%
}{}%
\DTLifeq{\status}{Attendee}{%
\DTLgadd{\attendee}{1}{\attendee}%
}{}%
}
\DTLsavelastrowcount{\totalconf}




\begin{cvhonors}

% \cvhonor{Organizer}{}{2020 -- Present}{\organizer{}}
% \cvhonor{Chair Positions}{}{2018 -- Present}{\chair{}}
% \cvhonor{Program Committee Member/Reviewer}{}{2018 -- Present}{\reviewer{}}
% \cvhonor{Community-Based Conference Attendee}{}{2013 -- Present}{\attendee{}}
% \cvhonor{Total Participation}{}{2013 -- Present}{\totalconf{}}

\cvhonor{Organizer}{}{}{\organizer{}}
\cvhonor{Chair Positions, including ACL Action Editor}{}{}{\chair{}}
\cvhonor{Program Committee Member/Reviewer}{}{}{\reviewer{}}
\cvhonor{Community-Based Conference Attendee}{}{}{\attendee{}}
\cvhonor{Total Participation}{}{}{\totalconf{}}

\end{cvhonors}

\vspace{0.7cm}

\cvsubsection{Organizer}


\begin{cventries}
\DTLforeach[\equal{\generalRole}{Organizer}]{organizing}{%
%ProjectName	Date	Role	Location	Description1	Description2    
\projectname=ConferenceName,%
\date=Date,%
\role=Role,%
\generalRole=GeneralRole,%
\location=Location,%
\link=link}%
{%
 \cventry
    {\role} % subhead
    {\href{\link}{\projectname}} % head
    {\date} % green
    {\location}% % flushright grey
    {%
    }
}
\end{cventries}


\cvsubsection{Other Chair Positions}


\begin{cventries}
\DTLforeach[\equal{\generalRole}{Chair}]{organizing}{%
\projectname=ConferenceName,%
\date=Date,%
\role=Role,%
\generalRole=GeneralRole,%
\location=Location,%
\link=link}%
{%
 \cventry
    {\role} % subhead
    {\projectname} % head
    {\date} % green
    {\location} % flushright grey
    {%
    }
}
\end{cventries}

\vspace{10pt}

\cvsubsection{Journal Reviewer}
\begin{cventries}
\DTLforeach[\equal{\role}{Journal Reviewer}]{organizing}{%
\projectname=ConferenceName,%
\date=Date,%
\role=Role,%
\generalRole=GeneralRole,%
\location=Location,%
\link=link}%
{%
    \begin{cvreviews}
    \cvreview
    {\projectname} % head
    {\date} % green
    \end{cvreviews}
}
\end{cventries}
\vspace{10pt}

\cvsubsection{ACL Rolling Review}
\begin{cventries}
\DTLforeach[\equal{\role}{ACL Rolling Review}]{organizing}{%
\projectname=ConferenceName,%
\date=Date,%
\role=Role,%
\generalRole=GeneralRole,%
\location=Location,%
\link=link,
\type=type}%
{%
    \begin{cvreviews}
      \cvreviewinfo
        {\generalRole}% role
        {\date}% Date(s)
        {\DTLifnullorempty{\type}{}{ -- \type}}%
    \end{cvreviews}
}
\end{cventries}
\vspace{10pt}


\cvsubsection{Program Committee Member}
\begin{cventries}
\DTLforeach[\equal{\role}{Program Committee Member}]{organizing}{%
\projectname=ConferenceName,%
\date=Date,%
\role=Role,%
\generalRole=GeneralRole,%
\location=Location,%
\link=link,
\type=type}%
{%
 \cventrysmallfont
    {\type} % subhead
    {\projectname} % head
    {\date} % green
    {\location} % flushright grey
    {}%
}
\end{cventries}





\cvsubsection{Community-Based Conference Attendance}
\begin{cventries}
\DTLforeach[\equal{\generalRole}{Attendee}]{organizing}{%
\projectname=ConferenceName,%
\date=Date,%
\role=Role,%
\generalRole=GeneralRole,%
\location=Location,%
\link=link}%
{%
    \begin{cvreviews}
    \cvreview
    {\projectname} % head
    {\date} % green
    \end{cvreviews}
}
\end{cventries}

\vspace{.5cm}
%-------------------------------------------------------------------------------
%	SECTION TITLE
%-------------------------------------------------------------------------------

{\color{black}\fontsize{12pt}{1em}\faIcon{hands-helping}} \cvsection{Mentorship}

\DTLloaddb[]{mentorship}{cv/mentoring.csv}
%-------------------------------------------------------------------------------
%	CONTENT
%-------------------------------------------------------------------------------

% \cvsubsection{University of Maryland, Baltimore County}
% \begin{cventries}
% \DTLforeach[\equal{\location}{UMBC}]{mentorship}{%
% \person=person,%
% \date=Date,%
% \location=location,%
% \then=Then,%
% \now=Now,
% \thesis=thesis}%
% {%
%  \cventry
%     {\then \DTLifnullorempty{\now}{}{ $\rightarrow$ \now}} % subhead
%     {\person \DTLifnullorempty{\thesis}{}{~(\href{\thesis}{thesis})}}% head
%     {\date}% green
%     {}% flushright grey
%     {}%
% }%
% \end{cventries}

% \hspace{2cm}




\cvsubsection{University of Maryland, Baltimore County}
\begin{cventries}
\DTLforeach[\equal{\location}{University of Maryland, Baltimore County}]{mentorship}{%
\person=person,%
\date=Date,%
\location=location,%
\then=Then,%
\now=Now,
\thesis=thesis}%
{%
 \cventry
    {\then \DTLifnullorempty{\now}{}{ $\rightarrow$ \now}} % subhead
    {\person \DTLifnullorempty{\thesis}{}{~(\href{\thesis}{thesis})}}% head
    {\date}% green
    {}% flushright grey
    {}%
}%
\end{cventries}


\hspace{2cm}

\cvsubsection{Outreach}
\begin{cventries}
 \cventry
    {Out in Tech U’s Mentorship Program} % subhead
    {Mark McGovern}% head
    {Summer 2022}% green
    {}% flushright grey
    {}%
\end{cventries}
\hspace{2cm}

\cvsubsection{University of Pennsylvania}
\begin{cventries}
\DTLforeach[\equal{\location}{University of Pennsylvania}]{mentorship}{%
\person=person,%
\date=Date,%
\location=location,%
\then=Then,%
\now=Now,
\thesis=thesis}%
{%
 \cventry
    {\then \DTLifnullorempty{\now}{}{ $\rightarrow$ \now}} % subhead
    {\person \DTLifnullorempty{\thesis}{}{~(\href{\thesis}{thesis})}}% head
    {\date}% green
    {}% flushright grey
    {}%
}%
\end{cventries}

\hspace{2cm}

\cvsubsection{Georgia Institute of Technology}

\begin{cventries}
\DTLforeach[\equal{\location}{Georgia Institute of Technology}]{mentorship}{%
\person=person,%
\date=Date,%
\location=location,%
\then=Then,%
\now=Now}%
{%
 \cventry
    {\then \DTLifnullorempty{\now}{}{ $\rightarrow$ \now}} % subhead
    {\person}% head
    {\date}% green
    {}% flushright grey
    {}%
}%
\end{cventries}

\vspace{.5cm}
%-------------------------------------------------------------------------------
%	SECTION TITLE

%-------------------------------------------------------------------------------

\cvsection{\faIcon{newspaper}}{Media Coverage}


%-------------------------------------------------------------------------------
%	SUBSECTION TITLE
%-------------------------------------------------------------------------------



%-------------------------------------------------------------------------------
%	CONTENT
%-------------------------------------------------------------------------------
%\begin{cvmediaenv}

%---------------------------------------------------------
  \cvmedia
    {``\href{https://youtu.be/K5cAzLGaDo8?si=sRf4fcP3jCiNWdDn}{Interactive Storytelling with AI – Dr. Lara Martin}''} % title & link
    {Jacob \& Warren Tingen} % author
    {Tingenuity AI (podcast)} % Location
    {Jul 11, 2025} % Date(s)

%---------------------------------------------------------
  \cvmedia
    {``\href{https://www.theregister.com/2023/08/19/chatgpt_dnd_dm/}{Hallucinating ChatGPT finds a role playing Dungeons \& Dragons}''} % title & link
    {Thomas Claburn} % author
    {The Register} % Location
    {Aug 19, 2023} % Date(s)

%---------------------------------------------------------
  \cvmedia
    {``\href{https://magazine.seas.upenn.edu/fall-2022/next-gen-innovators/}{Next-Gen Innovators: Penn Engineering Postdoctoral Fellows Lead the Way on Groundbreaking Research}''} % title & link
    {Amy Biemiller} % author
    {Penn Engineering Magazine} % Location
    {Fall 2022} % Date(s)

%---------------------------------------------------------
  \cvmedia
    {``\href{https://magazine.tank.tv/issue-88/talk/lara-martin}{Lara Martin: ``How can I get a system to tell a story about anything I want?''}''} % title & link
    {Masoud Golsorkhi} % author
    {TANK Magazine, Issue 88 (Narrative)} % Location
    {Autumn 2021} % Date(s)

%---------------------------------------------------------
  \cvmedia
    {``\href{https://www.sciencefocus.com/future-technology/lara-martin-on-teaching-ai-to-tell-stories/}{Lara Martin on teaching AI to tell stories}''} % title & link
    {Amy Barrett} % author
    {BBC Science Focus Podcast} % Location
    {Mar 15, 2021} % Date(s)

%---------------------------------------------------------
  \cvmedia
    {``\href{https://www.sciencefocus.com/magazine/dark-stars/}{Alexa, tell me a story}''} % Award
    {Amy Barrett} % author
    {BBC Science Focus Magazine} % Location
    {Feb 17, 2021} % Date(s)

%---------------------------------------------------------
  \cvmedia
    {``\href{https://web.archive.org/web/20230128071120/https://mlatgt.blog/2020/04/16/meet-mlgt-lara-j-martin-trains-ai-agents-to-become-storytellers/}{Meet ML$@$GT: Lara J. Martin Trains AI Agents to Become Storytellers}''} % title & link
    {Allie McFadden} % author
    {Georgia Tech Machine Learning, Memos from ML\@GT} % Location
    {Apr 16, 2020} % Date(s)

%---------------------------------------------------------
  \cvmedia
    {``\href{https://www.wired.com/story/forget-chess-real-challenge-teaching-ai-play-dandd/}{Forget Chess---the Real Challenge Is Teaching AI to Play D\&D}''} % title & link
    {Will Knight} % author
    {Wired} % Location
    {Feb 28, 2020} % Date(s)

%---------------------------------------------------------
  \cvmedia
    {``\href{https://www.gatech.edu/news/2020/02/04/changing-conversation-georgia-tech-researchers-provide-new-approach-automated-story}{Changing the Conversation: Georgia Tech Researchers Provide New Approach to Automated Story Generation}''} % title & link
    {David Mitchell} % author
    {Georgia Tech School of Interactive Computing} % Location
    {Feb 4, 2020} % Date(s)

%---------------------------------------------------------
  % \cvmedia
  %   {``\href{https://zeszytymaryny.pl/naukowo/sztuczna-inteligencja-jako-pisarz-generowanie-fabuly/}{Sztuczna inteligencja jako pisarz: Generowanie fabuły}'' \newline (Translation from Polish: Artificial Intelligence as a Writer: Story Generation)} % title & link
  %   {Patrycja Świeczkowska} % author
  %   {Zeszyty Maryny (Blog)} % Location
  %   {Oct 4, 2019} % Date(s)

%---------------------------------------------------------

  \cvmedia
    {``\href{https://spectrum.ieee.org/pictionary-playing-ai-sketches-the-future-of-human-machine-collaborations}{Pictionary-Playing AI Sketches the Future of Human-Machine Collaborations}'' (mention)} % title & link
    {Eliza Strickland} % author
    {IEEE Spectrum} % Location
    {Feb 6, 2019} % Date(s)

%---------------------------------------------------------

  \cvmedia
    {``\href{https://web.archive.org/web/20220512160546/https://gvu.gatech.edu/georgia-tech-aaai2018}{Georgia Tech Artificial Intelligence Research Includes Collaborative Approaches with Humans, Automating Content, and More}''} % title & link
    {Joshua Preston} % author
    {Georgia Tech GVU Center} % Location
    {Feb 2, 2018} % Date(s)

%---------------------------------------------------------

%   \cvmedia
%     {``\href{https://medium.com/@mark_riedl/improvisational-storytelling-in-open-world-2359d7f649cf}{Improvisational Computational Storytelling in Open Worlds}''} % Award
%     {Mark Riedl} % Event
%     {Medium} % Location
%     {Jul 24, 2017} % Date(s)

%---------------------------------------------------------
%\end{cvmediaenv}



% {\color{black}\fontsize{12pt}{1em}\faIcon{users}} \cvsection{Institutional Service}

\begin{cventries}
\cventry
    {Reviewer}
    {President's Undergraduate Research Awards (PURA)}
    {Summer 2019}
    {Georgia Institute of Technology}
    {}    
\cventry
    {Volunteer}
    {School of Interactive Computing’s Prospective Student Visit Week}
    {Spring '16, '17, '18}
    {Georgia Institute of Technology}
    {}
\cventry
    {Member}
    {School of Interactive Computing Faculty Hiring Committee}
    {Fall 2018}
    {Georgia Institute of Technology}
    {}
\cventry
    {Member}
    {Graduate Student Council}
    {Fall 2018 -- Spring 2019}
    {Georgia Institute of Technology}
    {}    
\cventry
    {Website Manager}
    {Human-Centered Computing Website}
    {Fall 2017 -- Spring 2019}
    {Georgia Institute of Technology}
    {}    
\cventry
    {Coordinator}
    {School of Interactive Computing's Prospective Student Visit Week}
    {Spring 2016}
    {Georgia Institute of Technology}
    {}
\end{cventries}



%\input{cv/skills.tex}
% %-------------------------------------------------------------------------------
%	SECTION TITLE
%-------------------------------------------------------------------------------
{\color{black}\fontsize{12pt}{1em}\faIcon{building}} \cvsection{Research Experience}


%-------------------------------------------------------------------------------
%	CONTENT
%-------------------------------------------------------------------------------
\begin{cventries}


%---------------------------------------------------------
  \cventry
    {University of Maryland, Baltimore County -- Computer Science and Electrical Engineering (CSEE) Department} % Organization
    {Assistant Professor} % Job title
    {Aug 2023 – Present} % Date(s)
    {Baltimore, MD} % Location
    {
      % \begin{cvitems} % Description(s) of tasks/responsibilities
      % \end{cvitems}
    }

%---------------------------------------------------------
  \cventry
    {University of Pennsylvania -- Computer and Information Science Department} % Organization
    {\href{https://cifellows2020.org/}{Computing Innovation Fellow (CIFellow)} Postdoctoral Researcher} % Job title
    {Jan 2021 – Aug 2023} % Date(s)
    {Philadelphia, PA} % Location
    {
      % \begin{cvitems} % Description(s) of tasks/responsibilities
      % \item {Identifying the story understanding capabilities of large language models (LLMs).}
      % \item {Developing a working AAC prototype given feedback from users.}
      % \item {Conducted semi-structured interviews with autistic adult users of augmentative and alternative communication (AAC).}
      % \end{cvitems}
    }
%---------------------------------------------------------
  \cventry
    {Georgia Institute of Technology -- School of Interactive Computing} % Organization
    {Graduate Research Assistant} % Job title
    {Aug 2015 – Dec 2020} % Date(s)
    {Atlanta, GA} % Location
    {
      % \begin{cvitems} % Description(s) of tasks/responsibilities
      %   \item {Created a complex end-to-end automated story generation pipeline.}
      % \end{cvitems}
    }

%---------------------------------------------------------
  \cventry
    {Amazon.com Inc. -- Alexa Smart Home Machine Learning} % Organization
    {Applied Scientist Intern} % Job title
    {May 2017 – Aug 2017} % Date(s)
    {Seattle, WA} % Location
    {
      % \begin{cvitems} % Description(s) of tasks/responsibilities
      %   \item {Identified potential research questions within Alexa Smart Home.}
      %   \item {Developed a system for identifying commands with an assumed context.}
      % \end{cvitems}
    }

%---------------------------------------------------------
  \cventry
   {Carnegie Mellon University -- Language Technologies Institute} % Organization
    {Graduate Research Assistant} % Job title
    {Sept 2013 – Aug 2015} % Date(s)
    {Pittsburgh, PA} % Location
    {
      % \begin{cvitems} % Description(s) of tasks/responsibilities
      %   \item {Created a zero-resource speech-to-speech translation system for the University of Pittsburgh Medical Center.}
      %   \item {Performed emotion recognition in noisy speech for event detection.}
      % \end{cvitems}
    }
%---------------------------------------------------------
  \cventry
   {University of Southern California -- Institute for Creative Technologies} % Organization
    {Intern} % Job title
    {May 2011 – Aug 2011} % Date(s)
    {Playa Vista, CA} % Location
    {
      % \begin{cvitems} % Description(s) of tasks/responsibilities
      %   \item {Wrote a chatbot for the Virtual Patient Project using Bruce Wilcox’s language Chatscript.}
      %   \item {Developed an authoring tool for the Chatscript language using Java.}
      %   \item {Designed and ran experiments comparing my Chatscript system to the project’s current chat system.}
      % \end{cvitems}
    }

%---------------------------------------------------------
\end{cventries}

% \input{cv/extracurricular.tex}
% \input{cv/honors.tex}
% \input{cv/presentation.tex}
% \input{cv/writing.tex}




\vfill
{\tiny
CV produced using \LaTeX. Modified version of code: \href{https://github.com/Denubis/Academic-LaTeX-CV-BibTeX-CSV}{\tt github.com/Denubis/Academic-LaTeX-CV-BibTeX-CSV}.}

%-------------------------------------------------------------------------------
\end{document}
